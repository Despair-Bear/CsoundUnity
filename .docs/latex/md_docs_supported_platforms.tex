\mbox{\hyperlink{class_csound_unity}{Csound\+Unity}} supports building for Windows, mac\+OS and Android 64bit (it has successfully been tested on Oculus Quest). The aim of \mbox{\hyperlink{class_csound_unity}{Csound\+Unity}} is to support other platforms in the future, starting from i\+OS and Web\+GL. 