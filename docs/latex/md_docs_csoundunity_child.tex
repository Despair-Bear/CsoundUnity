New in version 3.\+0, the \mbox{\hyperlink{class_csound_unity_child}{Csound\+Unity\+Child}} component lets you read the Audio\+Channels found in a \mbox{\hyperlink{class_csound_unity}{Csound\+Unity}} instance. You can have as many audio channels you want in your Csd. You can set them with the chnset opcode. See the example below\+: 
\begin{DoxyCode}{0}
\DoxyCodeLine{<Cabbage>}
\DoxyCodeLine{form caption("{}Test CsoundUnityChild"{}) }
\DoxyCodeLine{rslider channel("{}gain"{}), range(0, 1, .4, 1, .01), text("{}Gain"{})}
\DoxyCodeLine{rslider range(0, 1, 1, 1, 0.001), channel("{}hrm1"{})}
\DoxyCodeLine{rslider range(0, 1, 0, 1, 0.001), channel("{}hrm2"{})}
\DoxyCodeLine{rslider range(0, 1, 0, 1, 0.001), channel("{}hrm3"{})}
\DoxyCodeLine{rslider range(0, 1, 0, 1, 0.001), channel("{}hrm4"{})}
\DoxyCodeLine{</Cabbage>}
\DoxyCodeLine{<CsoundSynthesizer>}
\DoxyCodeLine{<CsOptions>}
\DoxyCodeLine{-\/n -\/d }
\DoxyCodeLine{</CsOptions>}
\DoxyCodeLine{<CsInstruments>}
\DoxyCodeLine{; Initialize the global variables. }
\DoxyCodeLine{ksmps = 32}
\DoxyCodeLine{nchnls = 2}
\DoxyCodeLine{0dbfs = 1}
\DoxyCodeLine{}
\DoxyCodeLine{giWave1 ftgen 1, 0, 4096, 10, 1}
\DoxyCodeLine{giWave2 ftgen 1, 0, 4096, 10, 1, .5, .25, .17}
\DoxyCodeLine{}
\DoxyCodeLine{;this instrument sends audio to two named channels}
\DoxyCodeLine{;this audio can be picked up by any CsoundUnityNode component..}
\DoxyCodeLine{instr ChildSounds}
\DoxyCodeLine{    a1 oscili 1, 440, giWave1}
\DoxyCodeLine{    chnset a1, "{}sound1"{}}
\DoxyCodeLine{    a2 oscili 1, 840, giWave1}
\DoxyCodeLine{    chnset a2, "{}sound2"{}}
\DoxyCodeLine{endin}
\DoxyCodeLine{}
\DoxyCodeLine{</CsInstruments>}
\DoxyCodeLine{<CsScore>}
\DoxyCodeLine{;causes Csound to run for about 7000 years...}
\DoxyCodeLine{f0 z}
\DoxyCodeLine{i"{}ChildSounds"{} 0 z}
\DoxyCodeLine{</CsScore>}
\DoxyCodeLine{</CsoundSynthesizer>}

\end{DoxyCode}


To be able to pick those named channels, create a Game\+Object and add the \mbox{\hyperlink{class_csound_unity_child}{Csound\+Unity\+Child}} component. An Audio\+Source will be added if there is none. There you can choose the Audio\+Channels that this \mbox{\hyperlink{class_csound_unity_child}{Csound\+Unity\+Child}} will play and in which configuration. If Audio\+Channels\+Setting is set to MONO, the selected Audio\+Channel will be played in both LEFT and RIGHT channel, at half volume, to have it perfectly centered. If Audio\+Channels\+Setting is set to STEREO, the selected Audio\+Channels will use the respective output channel.

\texorpdfstring{$<$}{<}img src=\char`\"{}images/setup\+Csound\+Unity\+Child\+\_\+v3.\+gif\char`\"{} alt=\char`\"{}\char`\"{}\mbox{\hyperlink{class_csound_unity_child}{Csound\+Unity\+Child}}\char`\"{}\char`\"{}/\texorpdfstring{$>$}{>} 